\section{Todo-list}

\subsection{Gene expression analysis and dimensionality reduction}
\begin{itemize}
    \item Gene expression data\\
    Explanation of single-cell RNA-seq data.
    \item Dimensionality reduction for visualization purposes\\
    Why it is hard to visualize $n$-dimenstional data. Also mention that there are approaches to reduce dimnesionality with some references to the next section. Build a bridge to the next section.
\end{itemize}

\subsection{Modeling data to Gaussian distributions}
\begin{itemize}
    \item \sout{Gaussian distribution}\\
    \sout{Short overview of the Gaussian distribution.}
    \item \sout{Latent variables}\\
    \sout{Give a small explanation of latent variables and some formula's for conditional probability in relation to Gaussian distributions.}
    \item Probabilistic PCA and EM\\
    This will probably be the longest section, explaining the basis of our model, including PPCA and the EM algorithm (both in general and in relation to PPCA). Formula's, workflow and algorithms are listed here. Conclude with a subsection for the benefits and drawbacks.
\end{itemize}

\subsection{Additions to the classical model}
\begin{itemize}
        \item \sout{Mixture models}\\
        \sout{Information on mixture models.}
        \item Hierarchy\\
        Mostly referring to Bishop's Hierarchy article. Small part bout the combination with mixture models.
        \item Model-specific additions\\
        This should refer to models like ZIFA and VPAC.
    \end{itemize}
    \subsection{TensorFlow Probability}
    Small section explaining TensorFlow Probability and mainly the fact that it will be used.

