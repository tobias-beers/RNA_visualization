\chapter{PPCA model defined in Stan}\label{AP:ppca_stan}

This below is an example of a PPCA model defined in Stan. Note that priors have been specified over the parameters $\bm{\mu}$, $\sigma^2$ and $\bm{W}$ (line 32). The distributions of $x$ and $z$ have been specified for the model to infer the log-likelihood by itself (line 39). The user could also have passed the log-likelihood to the model directly instead. The latent distribution is specified to have a covariance matrix $\bm{I}_m$, to make sure that the latent dimensions show little to no correlation.

\vspace{25px}

\begin{lstlisting}[caption={PCPA model defined in Stan}, label={lst:ppca}, language=Stan]
data{
    int<lower=0> N;// number  of  data-points
    int<lower=0> D;// number  of  dimensions  in  observed  data-set
    int<lower=0> M;// number  of  dimensions  in  latent  data-set
    vector[D] x[N];// observed data
}

transformed data{
    vector[M] mean_z;
    matrix[M,M] cov_z;
    
    for (m in 1:M){
        mean_z[m] = 0.0;
        for (n in 1:M){
            if (m==n){
                cov_z[m,n]=1.0;
            } else{
                cov_z[m,n]=0.0;
            }
        }
    }
}

parameters{
    matrix[M,N] z;          // latent data
    matrix[D,M] W;          // factor loadings
    real<lower=0> sigma;    // standard  deviations
    vector[D] mu;           // added means
}

model{
    // priors
    for (d in 1:D){
        W[d] ~ normal(0.0,sigma);
        mu[d]~normal(0.0, 5.0) ;
        }
    sigma~lognormal(0.0, 1.0) ;
    
    // likelihood
    for (n in 1:N){
        z[:,n] ~ multi_normal(mean_z, cov_z);
        x[n] ~ normal(W*col(z,n)+mu, sigma);
        }    
}
\end{lstlisting}